\section{Experience} \label{progress}
This section describes my personal experience with the different phases of the graduation assignment. It includes a short reflection on the process, the result, the schedule, and the knowledge I have gained.

\subsubsection{Subject}
I remember the first discussions with Lloyd on my daily challenges in product management. One of the discussions made me think about a situation in which the organisation made a decision that just did not seem to make sense at the time. Could we have influenced the outcome of the decision by using evidence? Could we have influenced the outcome of the decision if we brought more structure into the discussion? These questions drove me towards the combination of product management and decision-making. The initial literature study revealed that I was not the only one interested in this subject. This result suggested I was on the right track.

\subsubsection{Iterative approach}
I started with an $80/20$ focus on product management and decision-making. Lloyd directed in flipping this ratio and generalising my contribution focused on decision-making. Eventually, the product management aspects served as helpful validation scenarios. 

I honestly did not expect that the graduation assignment would be so iterative in its approach. I expected a \emph{waterfall} approach, in which the research preparation served as the design of the research and the graduation assignment executed that design. I eventually lost track of the number of changes in the main research question, methodology, and title. This approach allowed us to steer the research based on relevant learnings. The downside of this approach is that the schedule that I initially created was useless. The Gantt charts looked helpful in theory, but their content changed regularly. The approach made it very difficult to plan activities. When the approach is so agile, it might have been useful to also plan in an agile way using sprints of three weeks. Looking back at my process, this is what I did without knowing it at the time. I delivered a, more or less, useful version of the thesis document every three weeks.

\subsubsection{Validation}
The first 40 versions of this document included three scenarios to validate the decision design pattern: requirements prioritisation, software pricing, and alternative solution selection. When I noticed I needed over thirty pages to describe the first scenario, I knew this was not the right approach. I strategically reduced the length of the first scenario using Lloyd's guidance without sacrificing too much content. Additionally, we decided to remove the software pricing scenario from the scope. Requirements prioritisation and alternative solution selection seemed better suitable scenarios for a Computer Science student. I further reduced the length of alternative solution selection and focused on the difference of the alternative solution selection and requirements prioritisation decisions in the context of this study.

\subsubsection{Result}
We need to find ways to help decision-makers structure decision-relevant information and allow them to understand the quality of the information in a reasonable time. This challenge will require a continuous stream of research focusing on the information structure, the challenges we can address using this information, and the usability of large volumes of information. I think this study contributes to addressing this challenge. Other researchers can use the results to find ways to make evidence-based decision-making more concrete or can search for concrete tools to bring evidence-based decision-making closer to the decision-maker.

\subsubsection{Learnings}
I learned a lot from the process and the content of this study. The literature study revealed a lot of interesting materials related to decision-making and product management. This experience helps me to keep consuming scientific materials. The discussions with Lloyd forced me to change my thinking on several occasions. Exchanging thoughts is essential to improve the content and the process continuously. I would like to close this paragraph with the notion that I learned that plans change, and plans should change. They need to evolve based on what we learn.